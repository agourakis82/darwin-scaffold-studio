\documentclass[authoryear,preprint,review,12pt]{elsarticle}

\usepackage{hyperref}
\usepackage{graphicx}
\usepackage{booktabs}
\usepackage{listings}
\usepackage{xcolor}

% Code listing style
\lstset{
  language=Julia,
  basicstyle=\ttfamily\small,
  keywordstyle=\color{blue},
  commentstyle=\color{gray},
  stringstyle=\color{red},
  numbers=left,
  numberstyle=\tiny,
  frame=single,
  breaklines=true
}

\journal{SoftwareX}

\begin{document}

\begin{frontmatter}

\title{Darwin Scaffold Studio: An Open-Source Julia Platform for Tissue Engineering Scaffold Analysis}

\author[pucsp]{Demetrios Chiuratto Agourakis\corref{cor1}}
\ead{demetrios@agourakis.med.br}
\cortext[cor1]{Corresponding author}

\author[pucsp]{Moema Alencar Hausen}

\affiliation[pucsp]{organization={Biomaterials and Regenerative Medicine Program, Pontifical Catholic University of S\~ao Paulo (PUC-SP)},
            city={S\~ao Paulo},
            country={Brazil}}

\begin{abstract}
Darwin Scaffold Studio is an open-source Julia platform for analyzing tissue engineering scaffolds from MicroCT and SEM imaging data. The software computes structural metrics including porosity ($<$1\% error), pore size (14.1\% APE against manual ground truth), interconnectivity, and tortuosity. Validation against the PoreScript dataset (DOI: 10.5281/zenodo.5562953) demonstrates accuracy suitable for comparative scaffold analysis. The platform integrates biomedical ontologies from OBO Foundry and exports FAIR-compliant metadata. Darwin Scaffold Studio provides an accessible, reproducible alternative to proprietary analysis tools for tissue engineering research.
\end{abstract}

\begin{keyword}
tissue engineering \sep scaffold analysis \sep microCT \sep image processing \sep Julia \sep open source
\end{keyword}

\end{frontmatter}

%% ============================================
\section{Motivation and significance}
%% ============================================

Tissue engineering scaffolds require precise structural characterization to ensure optimal cell infiltration and tissue regeneration \cite{murphy2010}. Key metrics include porosity, pore size, interconnectivity, and tortuosity, which directly influence cell attachment, proliferation, and nutrient transport \cite{karageorgiou2005}.

Current analysis tools present significant barriers to researchers:
\begin{itemize}
\item Commercial software (CTAn, Avizo) costs \$5,000--15,000/year
\item Custom Python/MATLAB scripts lack validation and reproducibility
\item Existing open-source tools (BoneJ) require manual ImageJ workflows
\end{itemize}

Darwin Scaffold Studio addresses these gaps by providing a validated, open-source platform implemented in Julia for high performance and reproducibility. The software is particularly relevant for research groups in developing countries where commercial licenses are prohibitively expensive.

%% ============================================
\section{Software description}
%% ============================================

\subsection{Software architecture}

Darwin Scaffold Studio is implemented in Julia 1.10+ with a modular architecture:

\begin{verbatim}
DarwinScaffoldStudio/
  Core/       # Types, configuration
  MicroCT/    # Image processing, metrics
  Ontology/   # OBO Foundry integration
  Science/    # Topology, percolation
\end{verbatim}

\subsection{Software functionalities}

\textbf{Image Processing:} The platform loads MicroCT volumes (RAW, TIFF, NIfTI) and SEM images. Preprocessing includes denoising, normalization, and Otsu adaptive thresholding for segmentation.

\textbf{Metrics Computation:} Table~\ref{tab:metrics} summarizes available metrics and their validation status.

\begin{table}[h]
\centering
\caption{Scaffold metrics and validation status}
\label{tab:metrics}
\begin{tabular}{lll}
\toprule
Metric & Method & Validation \\
\midrule
Porosity & Voxel counting & $<$1\% error (synthetic) \\
Pore size & Connected components & 14.1\% APE (real data) \\
Interconnectivity & Largest component ratio & Theoretical \\
Tortuosity & Dijkstra shortest path & Theoretical \\
\bottomrule
\end{tabular}
\end{table}

\textbf{Ontology Integration:} The platform integrates 1,200+ terms from OBO Foundry ontologies (UBERON, CL, CHEBI), enabling lookup of optimal scaffold parameters by tissue type and FAIR-compliant metadata export.

%% ============================================
\section{Illustrative example}
%% ============================================

The following example generates a synthetic Gyroid scaffold and computes structural metrics:

\begin{lstlisting}
using Images, Statistics

# Generate 64^3 Gyroid scaffold
size = 64
volume = zeros(Bool, size, size, size)
for i in 1:size, j in 1:size, k in 1:size
    x, y, z = 2pi .* (i, j, k) ./ size
    gyroid = sin(x)*cos(y) + sin(y)*cos(z) + sin(z)*cos(x)
    volume[i,j,k] = gyroid > 0.3
end

# Compute porosity
porosity = 1 - sum(volume) / length(volume)
println("Porosity: $(round(porosity*100, digits=1))%")
# Output: Porosity: 59.8%
\end{lstlisting}

This example runs in under 60 seconds without external data dependencies, facilitating reproducibility verification.

%% ============================================
\section{Validation}
%% ============================================

\subsection{Synthetic ground truth}

Validation against TPMS surfaces (Gyroid, Schwarz P) with analytical properties shows $<$1\% error for porosity and surface area measurements.

\subsection{Experimental validation}

The pore size algorithm was validated against the PoreScript dataset \cite{porescript}, which contains SEM images of salt-leached scaffolds with manual measurements (n=374).

\begin{table}[h]
\centering
\caption{Validation against PoreScript ground truth}
\label{tab:validation}
\begin{tabular}{llll}
\toprule
Metric & Darwin & Ground Truth & APE \\
\midrule
Pore size & 149.4 $\mu$m & 174.0 $\mu$m & 14.1\% \\
\bottomrule
\end{tabular}
\end{table}

\textbf{Limitations:} Darwin systematically underestimates pore size by approximately 15\%. This bias should be considered when interpreting absolute measurements. The validation is limited to 3 SEM images; additional validation on diverse scaffolds is ongoing.

%% ============================================
\section{Impact}
%% ============================================

Darwin Scaffold Studio enables:
\begin{enumerate}
\item \textbf{Cost reduction:} Free alternative to commercial software
\item \textbf{Reproducibility:} Self-contained examples with no external dependencies
\item \textbf{Education:} Simple API suitable for graduate-level instruction
\item \textbf{Interoperability:} FAIR data export facilitates meta-analyses
\end{enumerate}

The software is currently used in ongoing research at PUC-SP for bioactive glass scaffold characterization.

%% ============================================
\section{Conclusions}
%% ============================================

Darwin Scaffold Studio provides an open-source, validated platform for tissue engineering scaffold analysis. The 14.1\% APE on pore size measurements is suitable for comparative studies. Future development includes 3D microCT validation against BoneJ and machine learning-based segmentation.

%% ============================================
%% Required metadata
%% ============================================

\section*{Current code version}

\begin{table}[h]
\centering
\begin{tabular}{ll}
\toprule
Code metadata & Description \\
\midrule
Current code version & v0.3.0 \\
Permanent link & \url{https://github.com/agourakis82/darwin-scaffold-studio} \\
Code Ocean compute capsule & -- \\
Legal Code License & MIT \\
Code versioning system & git \\
Software languages & Julia 1.10+ \\
Compilation requirements & None (interpreted) \\
Developer documentation & README.md, docs/ \\
Support email & demetrios@agourakis.med.br \\
\bottomrule
\end{tabular}
\end{table}

\section*{Acknowledgments}

The authors thank the PUC-SP Biomaterials and Regenerative Medicine Program. This work was supported by [funding information].

\section*{Declaration of competing interest}

The authors declare no competing interests.

\bibliographystyle{elsarticle-harv}
\begin{thebibliography}{9}

\bibitem[Murphy et al.(2010)]{murphy2010}
Murphy, C.M., Haugh, M.G., O'Brien, F.J., 2010. The effect of mean pore size on cell attachment, proliferation and migration in collagen-glycosaminoglycan scaffolds for bone tissue engineering. Biomaterials 31, 461--466.

\bibitem[Karageorgiou and Kaplan(2005)]{karageorgiou2005}
Karageorgiou, V., Kaplan, D., 2005. Porosity of 3D biomaterial scaffolds and osteogenesis. Biomaterials 26, 5474--5491.

\bibitem[Jenkins et al.(2021)]{porescript}
Jenkins, M.J., et al., 2021. PoreScript: Automated Pore Size Analysis. Zenodo. \url{https://doi.org/10.5281/zenodo.5562953}

\end{thebibliography}

\end{document}
